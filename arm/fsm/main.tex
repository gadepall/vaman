\documentclass[journal,12pt,twocolumn]{IEEEtran}
%
\usepackage{setspace}
\usepackage{gensymb}
%\doublespacing
\singlespacing

%\usepackage{graphicx}
%\usepackage{amssymb}
%\usepackage{relsize}
\usepackage[cmex10]{amsmath}
\usepackage{siunitx}
%\usepackage{amsthm}
%\interdisplaylinepenalty=2500
%\savesymbol{iint}
%\usepackage{txfonts}
%\restoresymbol{TXF}{iint}
%\usepackage{wasysym}
\usepackage{amsthm}
%\usepackage{iithtlc}
\usepackage{mathrsfs}
\usepackage{txfonts}
\usepackage{stfloats}
\usepackage{steinmetz}
\usepackage{supertabular}
%\usepackage{bm}
\usepackage{cite}
\usepackage{cases}
\usepackage{subfig}
%\usepackage{xtab}
\usepackage{longtable}
\usepackage{multirow}
%\usepackage{algorithm}
%\usepackage{algpseudocode}
\usepackage{enumitem}
\usepackage{mathtools}
\usepackage{tikz}
\usepackage{circuitikz}
\usepackage{verbatim}
\usepackage{tfrupee}
\usepackage[breaklinks=true]{hyperref}
%\usepackage{stmaryrd}
\usepackage{tkz-euclide} % loads  TikZ and tkz-base
%\usetkzobj{all}
\usetikzlibrary{calc,math}
\usetikzlibrary{fadings}
\usepackage{listings}
    \usepackage{color}                                            %%
    \usepackage{array}                                            %%
    \usepackage{longtable}                                        %%
    \usepackage{calc}                                             %%
    \usepackage{multirow}                                         %%
    \usepackage{hhline}                                           %%
    \usepackage{ifthen}                                           %%
  %optionally (for landscape tables embedded in another document): %%
    \usepackage{lscape}     
\usepackage{multicol}
\usepackage{chngcntr}
\usepackage{blkarray}
\usepackage{karnaugh-map}
\usepackage{fontspec}
\usepackage[intoc]{nomencl}
\makenomenclature

%\usetikzlibrary{arrows, shapes.gates.logic.US, calc}
\usetikzlibrary{arrows,shapes.gates.logic.US,shapes.gates.logic.IEC,calc}
\setmainfont[ Path = fonts/]{Sanskrit_2003.ttf}
%\setmainfont{Sanskrit_2003.ttf}
%\setmainfont{Nakula.ttf}
%\setmainfont{Lohit-Devanagari.ttf}


%\usepackage{enumerate}

%\usepackage{wasysym}
%\newcounter{MYtempeqncnt}
\DeclareMathOperator*{\Res}{Res}
%\renewcommand{\baselinestretch}{2}
\renewcommand\thesection{\arabic{section}}
\renewcommand\thesubsection{\thesection.\arabic{subsection}}
\renewcommand\thesubsubsection{\thesubsection.\arabic{subsubsection}}

\renewcommand\thesectiondis{\arabic{section}}
\renewcommand\thesubsectiondis{\thesectiondis.\arabic{subsection}}
\renewcommand\thesubsubsectiondis{\thesubsectiondis.\arabic{subsubsection}}

% correct bad hyphenation here
\hyphenation{op-tical net-works semi-conduc-tor}
\def\inputGnumericTable{}                                 %%


\lstset{
%language=shell,
%language = Prolog,
frame=single, 
breaklines=true,
%showstringspaces=false,
columns=fullflexible
literate = {-}{-}1
}
%\lstset{
%language=tex,
%frame=single, 
%breaklines=true
%}

\begin{document}
%


\newtheorem{theorem}{Theorem}[section]
\newtheorem{problem}{Problem}
\newtheorem{proposition}{Proposition}[section]
\newtheorem{lemma}{Lemma}[section]
\newtheorem{corollary}[theorem]{Corollary}
\newtheorem{example}{Example}[section]
\newtheorem{definition}[problem]{Definition}
%\newtheorem{thm}{Theorem}[section] 
%\newtheorem{defn}[thm]{Definition}
%\newtheorem{algorithm}{Algorithm}[section]
%\newtheorem{cor}{Corollary}
\newcommand{\BEQA}{\begin{eqnarray}}
\newcommand{\EEQA}{\end{eqnarray}}
\newcommand{\define}{\stackrel{\triangle}{=}}

\bibliographystyle{IEEEtran}
%\bibliographystyle{ieeetr}


\providecommand{\mbf}{\mathbf}
\providecommand{\pr}[1]{\ensuremath{\Pr\left(#1\right)}}
\providecommand{\qfunc}[1]{\ensuremath{Q\left(#1\right)}}
\providecommand{\sbrak}[1]{\ensuremath{{}\left[#1\right]}}
\providecommand{\lsbrak}[1]{\ensuremath{{}\left[#1\right.}}
\providecommand{\rsbrak}[1]{\ensuremath{{}\left.#1\right]}}
\providecommand{\brak}[1]{\ensuremath{\left(#1\right)}}
\providecommand{\lbrak}[1]{\ensuremath{\left(#1\right.}}
\providecommand{\rbrak}[1]{\ensuremath{\left.#1\right)}}
\providecommand{\cbrak}[1]{\ensuremath{\left\{#1\right\}}}
\providecommand{\lcbrak}[1]{\ensuremath{\left\{#1\right.}}
\providecommand{\rcbrak}[1]{\ensuremath{\left.#1\right\}}}
\providecommand{\ceil}[1]{\left \lceil #1 \right \rceil }
\theoremstyle{remark}
\newtheorem{rem}{Remark}
\newcommand{\sgn}{\mathop{\mathrm{sgn}}}
\providecommand{\abs}[1]{\left\vert#1\right\vert}
\providecommand{\res}[1]{\Res\displaylimits_{#1}} 
\providecommand{\norm}[1]{\left\lVert#1\right\rVert}
%\providecommand{\norm}[1]{\lVert#1\rVert}
\providecommand{\mtx}[1]{\mathbf{#1}}
\providecommand{\mean}[1]{E\left[ #1 \right]}
\providecommand{\fourier}{\overset{\mathcal{F}}{ \rightleftharpoons}}
%\providecommand{\hilbert}{\overset{\mathcal{H}}{ \rightleftharpoons}}
\providecommand{\system}{\overset{\mathcal{H}}{ \longleftrightarrow}}
	%\newcommand{\solution}[2]{\textbf{Solution:}{#1}}
\newcommand{\solution}{\noindent \textbf{Solution: }}
\newcommand{\cosec}{\,\text{cosec}\,}
\providecommand{\dec}[2]{\ensuremath{\overset{#1}{\underset{#2}{\gtrless}}}}
\newcommand{\myvec}[1]{\ensuremath{\begin{pmatrix}#1\end{pmatrix}}}
\newcommand{\mydet}[1]{\ensuremath{\begin{vmatrix}#1\end{vmatrix}}}
%\numberwithin{equation}{section}
\numberwithin{equation}{subsection}
%\numberwithin{problem}{section}
%\numberwithin{definition}{section}
\makeatletter
\@addtoreset{figure}{problem}
\makeatother

\let\StandardTheFigure\thefigure
\let\vec\mathbf
%\renewcommand{\thefigure}{\theproblem.\arabic{figure}}
%\renewcommand{\thefigure}{\theproblem}
\renewcommand{\thefigure}{\thesection}
%\setlist[enumerate,1]{before=\renewcommand\theequation{\theenumi.\arabic{equation}}
%\counterwithin{equation}{enumi}


%\renewcommand{\theequation}{\arabic{subsection}.\arabic{equation}}

\def\putbox#1#2#3{\makebox[0in][l]{\makebox[#1][l]{}\raisebox{\baselineskip}[0in][0in]{\raisebox{#2}[0in][0in]{#3}}}}
     \def\rightbox#1{\makebox[0in][r]{#1}}
     \def\centbox#1{\makebox[0in]{#1}}
     \def\topbox#1{\raisebox{-\baselineskip}[0in][0in]{#1}}
     \def\midbox#1{\raisebox{-0.5\baselineskip}[0in][0in]{#1}}

\vspace{3cm}

\title{
%	\logo{
Digital Design through Vaman
%	}
}
\author{ 
%गाड़ेपल्लि वेंकट विश्वनाथ शर्मा $^{*}$% <-this % stops a space
%	\thanks{*रचयिता भारतीय प्रौद्योगिकी संस्थान, हैदराबाद,५०२२८५ के विद्युत अभियान्त्रिकी विभाग में कार्यरत हैं, ईमेल:gadepall@ee.iith.ac.in। यह लेख मुक्त स्रोत विचारधारा के अनुरूप  है।}
G V V Sharma$^{*}$% <-this % stops a space
	\thanks{*The author is with the Department of Electrical Engineering, IIT Hyderabad, 502285. email:gadepall@ee.iith.ac.in. All content in this manual is released under GNU/GPL.}
	
}	

% \author{ गाड़ेपल्लि वेंकट विश्वनाथ शर्मा $^{*}$% <-this % stops a space
% 	\thanks{*रचयिता भारतीय प्रौद्योगिकी संस्थान, हैदराबाद,५०२२८५ के विद्युत अभियान्त्रिकी विभाग में कार्यरत हैं, ईमेल:gadepall@ee.iith.ac.in। यह लेख मुक्त स्रोत विचारधारा के अनुरूप  है।}
	
% }	
%\title{
%	\logo{Matrix Analysis through Octave}{\begin{center}\includegraphics[scale=.24]{tlc}\end{center}}{}{HAMDSP}
%}


% paper title
% can use linebreaks \\ within to get better formatting as desired
%\title{Matrix Analysis through Octave}
%
%
% author names and IEEE memberships
% note positions of commas and nonbreaking spaces ( ~ ) LaTeX will not break
% a structure at a ~ so this keeps an author's name from being broken across
% two lines.
% use \thanks{} to gain access to the first footnote area
% a separate \thanks must be used for each paragraph as LaTeX2e's \thanks
% was not built to handle multiple paragraphs
%

%\author{<-this % stops a space
%\thanks{}}
%}
% note the % following the last \IEEEmembership and also \thanks - 
% these prevent an unwanted space from occurring between the last author name
% and the end of the author line. i.e., if you had this:
% 
% \author{....lastname \thanks{...} \thanks{...} }
%                     ^------------^------------^----Do not want these spaces!
%
% a space would be appended to the last name and could cause every name on that
% line to be shifted left slightly. This is one of those "LaTeX things". For
% instance, "\textbf{A} \textbf{B}" will typeset as "A B" not "AB". To get
% "AB" then you have to do: "\textbf{A}\textbf{B}"
% \thanks is no different in this regard, so shield the last } of each \thanks
% that ends a line with a % and do not let a space in before the next \thanks.
% Spaces after \IEEEmembership other than the last one are OK (and needed) as
% you are supposed to have spaces between the names. For what it is worth,
% this is a minor point as most people would not even notice if the said evil
% space somehow managed to creep in.



% The paper headers
%\markboth{Journal of \LaTeX\ Class Files,~Vol.~6, No.~1, January~2007}%
%{Shell \MakeLowercase{\textit{et al.}}: Bare Demo of IEEEtran.cls for Journals}
% The only time the second header will appear is for the odd numbered pages
% after the title page when using the twoside option.
% 
% *** Note that you probably will NOT want to include the author's ***
% *** name in the headers of peer review papers.                   ***
% You can use \ifCLASSOPTIONpeerreview for conditional compilation here if
% you desire.




% If you want to put a publisher's ID mark on the page you can do it like
% this:
%\IEEEpubid{0000--0000/00\$00.00~\copyright~2007 IEEE}
% Remember, if you use this you must call \IEEEpubidadjcol in the second
% column for its text to clear the IEEEpubid mark.



% make the title area
\maketitle

\newpage

\tableofcontents


\bigskip

\renewcommand{\thefigure}{\theenumi}
\renewcommand{\thetable}{\theenumi}
% \renewcommand{\abstractname}{सार}
% \renewcommand{\nomname}{नामकरण}
% \renewcommand{\solution}{हल: }
% \renewcommand{\figurename}{आकृति.}
% \renewcommand{\tablename}{सारणी.}
%\renewcommand{\theequation}{\theenumi}

%\begin{abstract}
%%\boldmath
%In this letter, an algorithm for evaluating the exact analytical bit error rate  (BER)  for the piecewise linear (PL) combiner for  multiple relays is presented. Previous results were available only for upto three relays. The algorithm is unique in the sense that  the actual mathematical expressions, that are prohibitively large, need not be explicitly obtained. The diversity gain due to multiple relays is shown through plots of the analytical BER, well supported by simulations. 
%
%\end{abstract}
% IEEEtran.cls defaults to using nonbold math in the Abstract.
% This preserves the distinction between vectors and scalars. However,
% if the journal you are submitting to favors bold math in the abstract,
% then you can use LaTeX's standard command \boldmath at the very start
% of the abstract to achieve this. Many IEEE journals frown on math
% in the abstract anyway.

% Note that keywords are not normally used for peerreview papers.
%\begin{IEEEkeywords}
%Cooperative diversity, decode and forward, piecewise linear
%\end{IEEEkeywords}



% For peer review papers, you can put extra information on the cover
% page as needed:
% \ifCLASSOPTIONpeerreview
% \begin{center} \bfseries EDICS Category: 3-BBND \end{center}
% \fi
%
% For peerreview papers, this IEEEtran command inserts a page break and
% creates the second title. It will be ignored for other modes.
%\IEEEpeerreviewmaketitle

\begin{abstract}
  This document shows how to use the Vaman to design a decade counter using a finite state machine (FSM).
%इस आलेख में वामन को  दशक गणित्र के रूप में उपयोग करने का विधान प्रस्तुत है।



%ॐ श्री गणेशाय नमः॥
%\\
%\indent जय श्री राम।
%This manual provides a simple introduction to Digital Design.
\end{abstract}

%\section{Nomenclature}
% \printnomenclature[1.7in]
% \input{hindi/nomen.tex}
\section{Software}
%\section{तंत्रांश}
All codes in this document are available at the following links. 
%इस आलेख के समस्त क्रमादेश निम्न जालबन्धन में उपलब्ध हैं।
\begin{lstlisting}
https://github.com/gadepall/vaman/tree/master/arm/codes/decoders
https://github.com/gadepall/vaman/tree/master/arm/codes/fsm
\end{lstlisting}
%
\section{Setup}
\renewcommand{\theequation}{\theenumi}
\renewcommand{\thefigure}{\theenumi}
\begin{enumerate}[label=\thesection.\arabic*.,ref=\thesection.\theenumi]
\numberwithin{equation}{enumi}
\numberwithin{figure}{enumi}
\numberwithin{table}{enumi}



\item Execute Table  \ref{table:disp_dec} using the Vaman and a seven segment display. 
\\
\solution The outputs $a,b,c,d,e,f,g$ in Table  \ref{table:disp_dec}  are expressed in terms of the inputs  $A,B,C,D$ through the following
equations. 
%
\begin{align}
\label{eq:disp_a}
a &= AB^{\prime}C^{\prime}D^{\prime}+A^{\prime}B^{\prime}CD^{\prime}
\\
\label{eq:disp_b}
b &= AB^{\prime}CD^{\prime}+A^{\prime}BCD^{\prime}
\\
\label{eq:disp_c}
c &= {D}^{\prime}{C}^{\prime}B{A}^{\prime}
\\
d &= AB^{\prime}C^{\prime}D^{\prime}+A^{\prime}B^{\prime}CD^{\prime}+ABCD^{\prime}+AB^{\prime}C^{\prime}D
\label{eq:disp_d}
\\
e &= AB^{\prime}C^{\prime}D^{\prime}+ABC^{\prime}D^{\prime}+A^{\prime}B^{\prime}CD^{\prime}+AB^{\prime}CD^{\prime}
\nonumber \\
&\quad +ABCD^{\prime}+AB^{\prime}C^{\prime}D
\label{eq:disp_e}
\\
f &= AB^{\prime}C^{\prime}D^{\prime}+A^{\prime}BC^{\prime}D^{\prime}+ABC^{\prime}D^{\prime}+ABCD^{\prime}
\label{eq:disp_f}
\\
g &= A^{\prime}B^{\prime}C^{\prime}D^{\prime}+AB^{\prime}C^{\prime}D^{\prime}+ABCD^{\prime}
\label{eq:disp_g}
\end{align}
\begin{table}
\centering
\input{./tables/disp_dec.tex}
\caption{Truth table for the display decoder}
\label{table:disp_dec}
\end{table}
%
Now execute the following code.  
\begin{lstlisting}
codes/decoders/dispdec/main.c
\end{lstlisting}
%Connect $A, B, C, D$ को GND से योजित करें।  तत्पश्चात चरों के भिन्न संचय के लिये प्रदर्शी में प्राप्त अंकों को सारणी \ref{table:disp_dec}  से सत्यापित करें।
For different values of $A,B,C,D$, execute the above code to verify Table \ref{table:disp_dec}.

%\item सारणी \ref{table:counter_decoder} में एक परवर्ती निष्कूटक के गुणधर्म का उल्लेख है।  बूलीय समीकरणों के द्वारा $A,B,C,D$ को $W,X,Y,Z$ के व्यञ्जकों में व्यक्त करें।  इसके पश्चात वामन के द्वारा परवर्ती निष्कूटक को कार्यान्वयित करें। 

\item Table \ref{table:counter_decoder} shows the logic for the incrementing decoder.  Express $A,B,C,D$ in terms of $W,X,Y,Z$.
\label{prob:counter_decoder}
\begin{table}
\centering
\input{./tables/counter_decoder.tex}
\caption{Truth table for the incrementing decoder}
\label{table:counter_decoder}
\end{table}
\\
\solution The desired expressions are available below. 
\begin{align}
\label{eq:inc_A}
A &= W^{\prime}X^{\prime}Y^{\prime}Z^{\prime} + W^{\prime}XY^{\prime}Z^{\prime}
+W^{\prime}X^{\prime}YZ^{\prime}
\nonumber \\
 & \quad +W^{\prime}XYZ^{\prime}
+W^{\prime}X^{\prime}Y^{\prime}Z
\\
\label{eq:inc_B}
B &= WX^{\prime}Y^{\prime}Z^{\prime} + W^{\prime}XY^{\prime}Z^{\prime}
\nonumber \\ 
& \quad 
+WX^{\prime}YZ^{\prime}
+W^{\prime}XYZ^{\prime}
\\
\label{eq:inc_C}
C &= WXY^{\prime}Z^{\prime} + W^{\prime}X^{\prime}YZ^{\prime}
\nonumber \\ 
& \quad 
+WX^{\prime}YZ^{\prime}
+W^{\prime}XYZ^{\prime}
\\
D &= WXYZ^{\prime} + W^{\prime}X^{\prime}Y^{\prime}Z
\label{eq:inc_D}
\end{align}
Execute the following code.  You should see the next number displayed.
%निम्न क्रमादेश का चालयन करें। प्रदर्शी में परवर्ती अंक उत्पन्न होंगे।

\begin{lstlisting}
codes/decoders/incdec/main.c
\end{lstlisting}
%
% \item आकृति. \ref{fig:pin_sheet} में  वामन के समस्त कुशाव्यूह  प्रस्तुत हैं।  कुशाव्यूह J5 को  आकृति \ref{fig:sevenseg} में प्रदत्त सप्तांश प्रदर्शी के कुशों से सारणी \ref{table:vaman_dispdec} के द्वारा योजित करें। ध्यान  रहे कि COM एवं 3.3V के मध्य एक प्रतिरोधी अनिवार्य है।  तत्पश्चात निम्न क्रमादेश का चालयन  करें। 
%
\item Fig. \ref{fig:pin_sheet} shows the pin diagram for the Vaman.  Connect the pins in bank J5 to the seven segment display using
Fig. \ref{fig:sevenseg} and Table \ref{table:vaman_dispdec}.  Do not forget to put a resistor between COM and 3.3V.  Then execute the following code 
\begin{lstlisting}
codes/fsm/dispdec/main.c
\end{lstlisting}
%

\begin{table}[!ht]
\centering
\parbox{.45\columnwidth}{
%\resizebox{0.5\columnwidth}{!}{
\input{tables/pygmy_sevenseg.tex}
}
\hfill
\parbox{.45\columnwidth}{
%\resizebox{0.5\columnwidth}{!}{
%\input{tables/Vaman_dispdec.tex}
\input{tables/pygmy_dispdec.tex}
}
\caption{Pin connection between the Vaman and seven segment display}
\label{table:vaman_dispdec}
\end{table}

\begin{figure}[!ht]
\centering
\resizebox{\columnwidth}{!}{
\input{figs/sevenseg.tex}
}
\caption{Seven segment display. }
\label{fig:sevenseg}
\end{figure}


\begin{figure*}[!ht]
\centering
\includegraphics[width = \textwidth]{figs/pin_sheet.png}
\caption{Pin diagram}
\label{fig:pin_sheet}
\end{figure*}
%
\item Modify the above code to obtain a decade counter.
\end{enumerate}

%

\section{Decade Counter }
\renewcommand{\theequation}{\theenumi}
\renewcommand{\thefigure}{\theenumi}
\begin{enumerate}[label=\thesection.\arabic*.,ref=\thesection.\theenumi]
\numberwithin{equation}{enumi}
\numberwithin{figure}{enumi}
\numberwithin{table}{enumi}

\item Use the Vaman to implement all the decoders in Fig.  \ref{fig:vaman_decade_counter}.  Implement the delay using a flip flop.  This is an 
example of an FSM which is implemented using a sequential circuit.

% के आधार पर क्रमादेश \ref{prob:counter_decoder}
% में संशोधन करें एवं वामन के द्वारा दशक गणित्र का निर्माण करें।
\begin{figure}[!ht]
\centering
\resizebox{\columnwidth}{!}{
\input{figs/pygmy_decade_counter.tex}
}
\caption{Block diagram of a decade counter.}
\label{fig:vaman_decade_counter}
\end{figure}


\end{enumerate}
%

\end{document}


