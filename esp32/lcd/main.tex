
\let\negmedspace\undefined
\let\negthickspace\undefined
\documentclass[journal,12pt,twocolumn]{IEEEtran}
%
\usepackage{setspace}
\usepackage{gensymb}
\usepackage{xcolor}
\usepackage{caption}
%\usepackage{subcaption}
%\doublespacing
\singlespacing

%\usepackage{graphicx}
%\usepackage{amssymb}
%\usepackage{relsize}
\usepackage[cmex10]{amsmath}
\usepackage{mathtools}
%\usepackage{amsthm}
%\interdisplaylinepenalty=2500
%\savesymbol{iint}
%\usepackage{txfonts}
%\restoresymbol{TXF}{iint}
%\usepackage{wasysym}
\usepackage{amsthm}
\usepackage{mathrsfs}
\usepackage{txfonts}
\usepackage{stfloats}
\usepackage{cite}
\usepackage{cases}
\usepackage{subfig}
%\usepackage{xtab}
\usepackage{longtable}
\usepackage{multirow}
%\usepackage{algorithm}
%\usepackage{algpseudocode}
\usepackage{enumitem}
\usepackage{mathtools}

%\usepackage[framemethod=tikz]{mdframed}
\usepackage{listings}
\usepackage{listings}
    \usepackage[latin1]{inputenc}                                 %%
    \usepackage{color}                                            %%
    \usepackage{array}                                            %%
    \usepackage{longtable}                                        %%
    \usepackage{calc}                                             %%
    \usepackage{multirow}                                         %%
    \usepackage{hhline}                                           %%
    \usepackage{ifthen}                                           %%
  %optionally (for landscape tables embedded in another document): %%
    \usepackage{lscape}     


%\usepackage{iithtlc}
\usepackage{tikz}
\usepackage{circuitikz}

%\usepackage{stmaryrd}


%\usepackage{wasysym}
%\newcounter{MYtempeqncnt}
\DeclareMathOperator*{\Res}{Res}
%\renewcommand{\baselinestretch}{2}
\renewcommand\thesection{\arabic{section}}
\renewcommand\thesubsection{\thesection.\arabic{subsection}}
\renewcommand\thesubsubsection{\thesubsection.\arabic{subsubsection}}

\renewcommand\thesectiondis{\arabic{section}}
\renewcommand\thesubsectiondis{\thesectiondis.\arabic{subsection}}
\renewcommand\thesubsubsectiondis{\thesubsectiondis.\arabic{subsubsection}}

% correct bad hyphenation here
\hyphenation{op-tical net-works semi-conduc-tor}

\lstset{
language=C,
frame=single, 
breaklines=true
}

%\lstset{
	%%basicstyle=\small\ttfamily\bfseries,
	%%numberstyle=\small\ttfamily,
	%language=Octave,
	%backgroundcolor=\color{white},
	%%frame=single,
	%%keywordstyle=\bfseries,
	%%breaklines=true,
	%%showstringspaces=false,
	%%xleftmargin=-10mm,
	%%aboveskip=-1mm,
	%%belowskip=0mm
%}

%\surroundwithmdframed[width=\columnwidth]{lstlisting}
\def\inputGnumericTable{}                                 %%
\lstset{
language=C,
frame=single, 
breaklines=true
}
 

\begin{document}
%

\theoremstyle{definition}
\newtheorem{theorem}{Theorem}[section]
\newtheorem{problem}{Problem}
\newtheorem{proposition}{Proposition}[section]
\newtheorem{lemma}{Lemma}[section]
\newtheorem{corollary}[theorem]{Corollary}
\newtheorem{example}{Example}[section]
\newtheorem{definition}{Definition}[section]
%\newtheorem{algorithm}{Algorithm}[section]
%\newtheorem{cor}{Corollary}
\newcommand{\BEQA}{\begin{eqnarray}}
\newcommand{\EEQA}{\end{eqnarray}}
\newcommand{\define}{\stackrel{\triangle}{=}}

\bibliographystyle{IEEEtran}
%\bibliographystyle{ieeetr}

\providecommand{\nCr}[2]{\,^{#1}C_{#2}} % nCr
\providecommand{\nPr}[2]{\,^{#1}P_{#2}} % nPr
\providecommand{\mbf}{\mathbf}
\providecommand{\pr}[1]{\ensuremath{\Pr\left(#1\right)}}
\providecommand{\qfunc}[1]{\ensuremath{Q\left(#1\right)}}
\providecommand{\sbrak}[1]{\ensuremath{{}\left[#1\right]}}
\providecommand{\lsbrak}[1]{\ensuremath{{}\left[#1\right.}}
\providecommand{\rsbrak}[1]{\ensuremath{{}\left.#1\right]}}
\providecommand{\brak}[1]{\ensuremath{\left(#1\right)}}
\providecommand{\lbrak}[1]{\ensuremath{\left(#1\right.}}
\providecommand{\rbrak}[1]{\ensuremath{\left.#1\right)}}
\providecommand{\cbrak}[1]{\ensuremath{\left\{#1\right\}}}
\providecommand{\lcbrak}[1]{\ensuremath{\left\{#1\right.}}
\providecommand{\rcbrak}[1]{\ensuremath{\left.#1\right\}}}
\theoremstyle{remark}
\newtheorem{rem}{Remark}
\newcommand{\sgn}{\mathop{\mathrm{sgn}}}
\providecommand{\abs}[1]{\left\vert#1\right\vert}
\providecommand{\res}[1]{\Res\displaylimits_{#1}} 
\providecommand{\norm}[1]{\lVert#1\rVert}
\providecommand{\mtx}[1]{\mathbf{#1}}
\providecommand{\mean}[1]{E\left[ #1 \right]}
\providecommand{\fourier}{\overset{\mathcal{F}}{ \rightleftharpoons}}
%\providecommand{\hilbert}{\overset{\mathcal{H}}{ \rightleftharpoons}}
\providecommand{\system}{\overset{\mathcal{H}}{ \longleftrightarrow}}
	%\newcommand{\solution}[2]{\textbf{Solution:}{#1}}
\newcommand{\solution}{\noindent \textbf{Solution: }}
\providecommand{\dec}[2]{\ensuremath{\overset{#1}{\underset{#2}{\gtrless}}}}
%\numberwithin{equation}{subsection}
\numberwithin{equation}{problem}
%\numberwithin{problem}{subsection}
%\numberwithin{definition}{subsection}
\makeatletter
\@addtoreset{figure}{problem}
\makeatother

\let\StandardTheFigure\thefigure
%\renewcommand{\thefigure}{\theproblem.\arabic{figure}}
\renewcommand{\thefigure}{\theproblem}


%\numberwithin{figure}{subsection}

%\numberwithin{equation}{subsection}
%\numberwithin{equation}{section}
%%\numberwithin{equation}{problem}
%%\numberwithin{problem}{subsection}
%\numberwithin{problem}{section}
%%\numberwithin{definition}{subsection}
%\makeatletter
%\@addtoreset{figure}{problem}
%\makeatother
%\makeatletter
%\@addtoreset{table}{problem}
%\makeatother

%\let\StandardTheFigure\thefigure
%\let\StandardTheTable\thetable
%%\renewcommand{\thefigure}{\theproblem.\arabic{figure}}
%\renewcommand{\thefigure}{\theproblem}
%\renewcommand{\thetable}{\theproblem}
%%\numberwithin{figure}{section}

%%\numberwithin{figure}{subsection}
\numberwithin{table}{subsection}


\def\putbox#1#2#3{\makebox[0in][l]{\makebox[#1][l]{}\raisebox{\baselineskip}[0in][0in]{\raisebox{#2}[0in][0in]{#3}}}}
     \def\rightbox#1{\makebox[0in][r]{#1}}
     \def\centbox#1{\makebox[0in]{#1}}
     \def\topbox#1{\raisebox{-\baselineskip}[0in][0in]{#1}}
     \def\midbox#1{\raisebox{-0.5\baselineskip}[0in][0in]{#1}}

\vspace{3cm}

\title{ 
%	\logo{
	Measuring Unkown Resistance Using ESP32
%	}
}
%\title{
%	\logo{Matrix Analysis through Octave}{\begin{center}\includegraphics[scale=.24]{tlc}\end{center}}{}{HAMDSP}
%}


% paper title
% can use linebreaks \\ within to get better formatting as desired
%\title{Matrix Analysis through Octave}
%
%
% author names and IEEE memberships
% note positions of commas and nonbreaking spaces ( ~ ) LaTeX will not break
% a structure at a ~ so this keeps an author's name from being broken across
% two lines.
% use \thanks{} to gain access to the first footnote area
% a separate \thanks must be used for each paragraph as LaTeX2e's \thanks
% was not built to handle multiple paragraphs
%

\author{G V V Sharma$^{*}$ %<-this  stops a space
\thanks{*The author is with the Department
of Electrical Engineering, Indian Institute of Technology, Hyderabad
502285 India e-mail:  gadepall@iith.ac.in. All content in the manual released under GNU GPL.  Free to use for anything.}% <-this % stops a space
%\thanks{J. Doe and J. Doe are with Anonymous University.}% <-this % stops a space
%\thanks{Manuscript received April 19, 2005; revised January 11, 2007.}}
}
% note the % following the last \IEEEmembership and also \thanks - 
% these prevent an unwanted space from occurring between the last author name
% and the end of the author line. i.e., if you had this:
% 
% \author{....lastname \thanks{...} \thanks{...} }
%                     ^------------^------------^----Do not want these spaces!
%
% a space would be appended to the last name and could cause every name on that
% line to be shifted left slightly. This is one of those "LaTeX things". For
% instance, "\textbf{A} \textbf{B}" will typeset as "A B" not "AB". To get
% "AB" then you have to do: "\textbf{A}\textbf{B}"
% \thanks is no different in this regard, so shield the last } of each \thanks
% that ends a line with a % and do not let a space in before the next \thanks.
% Spaces after \IEEEmembership other than the last one are OK (and needed) as
% you are supposed to have spaces between the names. For what it is worth,
% this is a minor point as most people would not even notice if the said evil
% space somehow managed to creep in.



% The paper headers
%\markboth{Journal of \LaTeX\ Class Files,~Vol.~6, No.~1, January~2007}%
%{Shell \MakeLowercase{\textit{et al.}}: Bare Demo of IEEEtran.cls for Journals}
% The only time the second header will appear is for the odd numbered pages
% after the title page when using the twoside option.
% 
% *** Note that you probably will NOT want to include the author's ***
% *** name in the headers of peer review papers.                   ***
% You can use \ifCLASSOPTIONpeerreview for conditional compilation here if
% you desire.




% If you want to put a publisher's ID mark on the page you can do it like
% this:
%\IEEEpubid{0000--0000/00\$00.00~\copyright~2007 IEEE}
% Remember, if you use this you must call \IEEEpubidadjcol in the second
% column for its text to clear the IEEEpubid mark.



% make the title area
\maketitle

%\newpage

%\tableofcontents


%\begin{abstract}
%%\boldmath
%In this letter, an algorithm for evaluating the exact analytical bit error rate  (BER)  for the piecewise linear (PL) combiner for  multiple relays is presented. Previous results were available only for upto three relays. The algorithm is unique in the sense that  the actual mathematical expressions, that are prohibitively large, need not be explicitly obtained. The diversity gain due to multiple relays is shown through plots of the analytical BER, well supported by simulations. 
%
%\end{abstract}
% IEEEtran.cls defaults to using nonbold math in the Abstract.
% This preserves the distinction between vectors and scalars. However,
% if the journal you are submitting to favors bold math in the abstract,
% then you can use LaTeX's standard command \boldmath at the very start
% of the abstract to achieve this. Many IEEE journals frown on math
% in the abstract anyway.

% Note that keywords are not normally used for peerreview papers.
%\begin{IEEEkeywords}
%Cooperative diversity, decode and forward, piecewise linear
%\end{IEEEkeywords}



% For peer review papers, you can put extra information on the cover
% page as needed:
% \ifCLASSOPTIONpeerreview
% \begin{center} \bfseries EDICS Category: 3-BBND \end{center}
% \fi
%
% For peerreview papers, this IEEEtran command inserts a page break and
% creates the second title. It will be ignored for other modes.
\IEEEpeerreviewmaketitle


%\documentclass{article}
%\usepackage[utf8]{inputenc}
%\usepackage{listings}
%\usepackage{graphicx}

%\usepackage{color}
%\definecolor{codegreen}{rgb}{0,0.6,0}
%\definecolor{codegray}{rgb}{0.5,0.5,0.5}
%\definecolor{codepurple}{rgb}{0.58,0,0.82}
%\definecolor{backcolour}{rgb}{0.95,0.95,0.92}
%\lstdefinestyle{mystyle}{
    %backgroundcolor=\color{backcolour},   
    %commentstyle=\color{codegreen},
    %keywordstyle=\color{magenta},
    %numberstyle=\tiny\color{codegray},
    %stringstyle=\color{codepurple},
    %basicstyle=\footnotesize,
    %breakatwhitespace=false,         
    %breaklines=true,                 
    %captionpos=b,                    
    %keepspaces=true,                 
    %numbers=left,                    
    %numbersep=5pt,                  
    %showspaces=false,                
    %showstringspaces=false,
    %showtabs=false,                  
    %tabsize=2
%}
 
%\lstset{style=mystyle}


%\title{Analog Design Through ESP32}
%\author{G V V Sharma* }

%\begin{document}

%\maketitle
\begin{abstract}
Through this manual, we learn how to measure an unknown resistance through ESP32 and display it on an LCD.
\end{abstract}
\section{Components}
\numberwithin{equation}{enumi}
\numberwithin{figure}{enumi}
%\numberwithin{table}{enumi}
\numberwithin{table}{section}


\begin{table}[!h]
\centering
\input{./figs/components7.tex}
\caption{}
\label{table:components}
\end{table}
\section{Setting up the Display}
\begin{enumerate}[label=\thesection.\arabic*.,ref=\thesection.\theenumi]
\numberwithin{equation}{enumi}
\numberwithin{figure}{enumi}
\numberwithin{table}{enumi}
\item
Plug the LCD in Fig. \ref{fig:lcd} to the breadboard.

%
\begin{figure}
\centering
\includegraphics[width=\columnwidth]{./figs/lcd.eps}
\caption{lcd}
\label{fig:lcd}
\end{figure}

%
%\item
%Connect the 220$\Omega$ resistance from $V_{cc}$ to pin 15 (Led+) of the LCD.

%
%%
%\item
%Connect  pin 16 (Led-) of the LCD to GND.  The LCD should glow.
%
%%
%\item
%Connect pin 3 of the LCD to GND.  This is required for contrast.
%
\item
Connect the ESP32 pins to LCD pins as per Table \ref{Table:1}. Make sure that all 5V sources are connected to the LCD through a 220 $\Omega$ resistance.
\begin{table}[h]
\centering
\input{./figs/table1.tex}
\caption{Make sure that all 5V sources are connected to the LCD through a 220 $\Omega$ resistance.}
\label{Table:1}
\end{table}

\item Execute the following code after editing the wifi credentials

\begin{lstlisting}
https://github.com/gadepall/vaman/tree/master/esp32/codes/ide/lcd/setup
\end{lstlisting}
You should see the following  message
\begin{lstlisting}
Hi
This is CSP Lab
\end{lstlisting}
\item Modify the above code to display your name.
\end{enumerate}
\section{Measuring the resistance}
\begin{enumerate}[label=\thesection.\arabic*.,ref=\thesection.\theenumi]
\numberwithin{equation}{enumi}
\numberwithin{figure}{enumi}
\numberwithin{table}{enumi}

\item
Connect the 5V pin of the ESP32 to an extreme pin of the Breadboard shown in Fig. \ref{fig:breadboard}.  Let this pin be $V_{cc}$.

%
%
\begin{figure}
\centering
\includegraphics[width=\columnwidth]{./figs/breadboard.eps}
\caption{Breadboard}
\label{fig:breadboard}
\end{figure}
%
\item
Connect the GND pin of the ESP32 to the opposite extreme pin of the Breadboard.

%
%
\item
Let $R_1$ be the known resistor and $R_2$ be the unknown resistor.  Connect $R_1$ and $R_2$ in series such that $R_1$ is connected
to $V_{cc}$ and $R_2$ is connected to GND. Refer to Fig. \ref{fig:voltage_divider}

%
%
\begin{figure}
\centering
%\includegraphics[width=\columnwidth]{./figs/voltage_divider.eps}
\resizebox {\columnwidth} {!} {
%\input{./figs/collpits.tex}
\input{./figs/vrr.tex}
}
%\input{./figs/vrr.tex}
%\includegraphics[width=\columnwidth]{./figs/voltage_divider.eps}
\caption{Voltage Divider}
\label{fig:voltage_divider}
\end{figure}
%
\item
Connect the junction between the two resistors to  the GPIO34 pin on the ESP32.

%
\item
Connect the ESP32 to the computer so that it is powered.
%
%
%
%\item
%The resistance is an analogue function,but the value displayed on LCD is digital function.So,we need to do analogue to digital conversion,ESP32 has built-in 10-bit analogue to digital converter.
%Two resistors are used, one of which is known resistor value and other is unknown resistor value.
%Take 2 resistors,one known resistor with resistance value(For e.g-1K,2K,10K),here we have taken 1K and another an unknown resistor whose value we are going to calculate.
%
%
%

\item
Execute the following code after editing the wifi credentials

\begin{lstlisting}
https://github.com/gadepall/vaman/tree/master/esp32/codes/ide/lcd/resistance
\end{lstlisting}
%\solution 
%%
%\lstinputlisting{./codes/ESP32 _LCD/src/main.cpp}

\end{enumerate}
\section{Explanation}
\begin{enumerate}[label=\thesection.\arabic*.,ref=\thesection.\theenumi]
\numberwithin{equation}{enumi}
\numberwithin{figure}{enumi}
\numberwithin{table}{enumi}

%\begin{enumerate}

\item We create a variable called analogPin and assign it to 0. 
This is because the voltage value we are going to read is connected to analogPin GPIO34.

\item  The 12-bit ADC can differentiate 4096 discrete voltage levels, 5 volt is applied to 2 resistors and the voltage sample is taken in between the resistors. The value which we get from analogPin can be between 0 and 4095. 0 would represent 0 volts falls across the unknown resistor. A value of 4095 would mean that practically all 5 volts falls across the unknown resistor.

\item  $V_{out}$ represents the divided voltage that falls across the unknown resistor.

\item  The Ohm meter in this manual works on the principle of the voltage divider shown in Fig. \ref{fig:voltage_divider}.
%
\begin{align}
V_{out}&=\frac{R_1}{R_1+R_2}V_{in} \\
\Rightarrow R_2&=R_1\brak{\frac{V_{in}}{V_{out}}-1}
\end{align}
%
In the above, $V_{in} = 5$V, $R_1 = 220 \Omega$.
\item Repeat the exercise with another unknown resistance.
\end{enumerate}


\end{document}


